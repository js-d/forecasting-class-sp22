\documentclass[11pt]{article}
\usepackage{amsmath,amsfonts,amssymb}
\usepackage{hyperref}
\hypersetup{
	colorlinks=true,
	linkcolor=blue,
	urlcolor=cyan
	}
\usepackage[margin=1in]{geometry}
\usepackage{graphicx}

\setlength{\parindent}{0pc}
\setlength{\parskip}{10pt}

\title{STAT157 HW 8}
\date{March 8, 2022}

\begin{document}

\maketitle

\hfill \textbf{Predictions due Thursday, March 17 at 11:59pm}

\hfill \textbf{Gradescope submission due 15 min after}


\section*{Deliberate Practice: Information Hygiene}

\emph{Expected completion time: 60 minutes}

Read a new article, blog post, or other kind of media that you would normally read and analyze the ``lifetime" of the information presented in that media, with the following components:
\begin{enumerate}
	\item Creation: How was the information in the piece created? Some example questions to ask yourself: If there was an experiment or survey, what was the methodology? If there was an interview, how were the interviewees selected and why might they have agreed to do an interview? 
	\item Transfer: How did this information find its way to me? Some example questions to ask yourself: Does my newsfeed filter out certain articles? Did a friend share this article, and why? Did the editor of the piece try to make the headline more attention-worthy, or otherwise affect whether the piece would make it onto my reading list?
	\item Digestion: How did I handle this information, after reading it? Some example questions to ask yourself: Did I accept the information at face value? Did I dismiss it immediately? Did I want to look up more evidence, either supporting the information or not? How would I expect someone I disagree with to react to this information?
	\item Writeup: In your writeup, include the title of the piece you read and your analysis of the above three components. Also include your overall takeaways from this exercise.
\end{enumerate}

On Gradescope, please also submit the time it took to complete this exercise.


\section*{Predictions}

\emph{Expected completion time: 60 minutes}

Register the following predictions. You can submit them by going to TODO\href{URL}{this URL} and following the form's instructions. For these predictions, (and all predictions about the future throughout this class), we encourage you to use external sources -- by googling things, reading news articles, talking to friends who follow politics or music stats, etc.

\begin{enumerate}
	\item 
	\item 
\end{enumerate}

For each question, submit a mean and inclusive 80\% confidence interval or your probabilities, as well as an explanation of your reasoning (1-2 paragraphs). \textbf{Please include a copy of your google form responses with your Gradescope submission.} On Gradescope, please also submit the time it took to complete this exercise.

\section*{Extra Credit}

You can earn extra credit for doing one or both of the following tasks:
\begin{enumerate}
\item (2 points). Propose a prediction question for a future homework. Your question should have a fully specified resolution criteria, 
      such that it would be clear to the course staff how to resolve it.
\item (2 points). Write down 5 ``calibration-app'' style questions (and answers), for a domain that is not trivia.
\end{enumerate}

For the first part, you will get extra credit if we either use your question on a future homework, 
or like it enough to add it to our question bank for future classes.

For the second part, you will get extra credit if we think they are at least as interesting as the 
questions currently in the calibration app, and well-written enough to be used as a calibration question. 
(We may submit them to the app creator and suggest they include them!)

\end{document}
