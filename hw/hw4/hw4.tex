\documentclass[11pt]{article}
\usepackage{amsmath,amsfonts,amssymb}
\usepackage{hyperref}
\hypersetup{
	colorlinks=true,
	linkcolor=blue,
	urlcolor=cyan
	}
\usepackage[margin=1in]{geometry}
\usepackage{graphicx}

\setlength{\parindent}{0pc}
\setlength{\parskip}{10pt}

\title{STAT157 HW 4}
\date{Jan 7, 2022}

\begin{document}

\maketitle

\hfill \textbf{Due Monday, February 14 at 11:59pm}

\section*{Deliberate Practice: ??}

\emph{Expected completion time: ?? minutes}

\section*{Lab: ??}

\emph{Expected completion time: ?? minutes}

\section*{Predictions}

\emph{Expected completion time: 60 minutes}

Register the following predictions. You can submit them by going to \url{https://forms.gle/q9RccKemxaRtrZ6o6} and following the form's instructions. For these predictions, (and all predictions about the future throughout this class), we encourage you to use external sources -- by googling things, reading news articles, talking to friends who follow politics or music stats, etc.

\begin{enumerate}
	\item[0.] Pick the same website, application, or software you chose last week, and predict how much time you will spend on it between 12:00am Tuesday February 15th and 11:59pm Sunday February 20th., as measured by your time-tracking app.

	\item[1.] How many NYT articles will there be about Russia-Ukraine tensions between Monday Feb 14 11:59pm and Monday Feb 21 11:59pm? This resolves as the number of articles published between both dates in the ``Latest'' list at \href{https://www.nytimes.com/news-event/ukraine-russia?name=styln-russia-ukraine&region=TOP_BANNER&block=storyline_menu_recirc&action=click&pgtype=LegacyCollection&variant=0_Control}{this link}.
	
	\item[2.] Will David Campos get the most votes in the Feb 15 Special primary election for California State Assembly District 17?
 
	\item[3.] Will the $\#1$ most sold Fiction book on Amazon for the week of February 13 be by Colleen Hoover? This resolves based on the Most Sold Fiction category in \href{https://www.amazon.com/charts/}{Amazon Charts} for the week of February 13.
\end{enumerate}

For each question, submit a mean and inclusive 80\% confidence interval (or a probability for questions 2 and 3), as well as an explanation of your reasoning (~1-2 paragraphs). \textbf{Please include a copy of your google form responses with your Gradescope submission.} On Gradescope, please also submit the time it took to complete this exercise.


\end{document}