\documentclass[11pt]{article}
\usepackage{amsmath,amsfonts,amssymb}
\usepackage{hyperref}
\hypersetup{
	colorlinks=true,
	linkcolor=blue,
	urlcolor=cyan
	}
\usepackage[margin=1in]{geometry}
\usepackage{graphicx}

\setlength{\parindent}{0pc}
\setlength{\parskip}{10pt}

\title{STAT157 HW 5}
\date{Feb 15, 2022}

\begin{document}

\maketitle

\hfill \textbf{Due Monday, February 21 at 11:59pm}

\section*{Deliberate Practice: Combining Forecasts}

\emph{Expected completion time: 90 minutes}


For the following questions, look up and provide links to at least 3 data sources that could be good reference classes. For each reference class, comment on: 
\begin{itemize}
	\item How closely related it is to the question of interest
	\item Whether or not you expect it to systematically over/under-estimate the answer
	\item How much error the reference class's finite sample size induces
	\item How independent the different reference classes are.
\end{itemize}

Use these considerations to assign relative weights to each reference class, and compute a weighted average. Some of the 3 data sources can be external predictions (e.g. on odds on a betting site), but at least one of the three should be an actual dataset.

\begin{enumerate}
	\item Will this year's NBA MVP be a guard?
	\item Will Trump file for president before the end of 2022?
	\item Will the Senate pass the \href{https://www.congress.gov/bill/117th-congress/senate-bill/3538}{EARN It Act of 2022}?
	\item (Extra credit) At what valuation will Discord IPO, assuming it does IPO? Unlike the previous questions and the example studied in the lecture, here you are trying to predict a numerical value: we will give up to 2 points of extra credit based on how you adapt the lecture's method.
\end{enumerate}

On Gradescope, please also submit the time it took to complete this exercise.

\section*{Lab}

\emph{Expected completion time: 120 minutes}

\href{https://bit.ly/34T4SlQ}{Link to Jupyter notebook.}

Please follow the instructions in the notebook to print out your code and answers and submit to Gradescope. You may use languages other than Python, although we will generally be providing starter code in Python.

On Gradescope, please also submit the time it took to complete this exercise.

\section*{Short-term Predictions}

\emph{Expected completion time: 60 minutes}

Register the following predictions. You can submit them by going to \url{https://forms.gle/hDsAcMvyMP7CJm1C7} and following the form's instructions. For these predictions, (and all predictions about the future throughout this class), we encourage you to use external sources -- by googling things, reading news articles, talking to friends who follow politics or music stats, etc.

\begin{enumerate}
	\item[1.] Will masks be mandatory to attend lectures in this class (STAT157/260) on March 8th?
	\begin{itemize}
		\item Note that the question is about attending lectures, not discussion sections or office hours.
		\item This question resolves positively if masks are mandatory to attend lectures (whether due to UC Berkeley policy, professor choice, or a vote by the students), unless this \emph{explicitly} contradicts UC Berkeley policy.
	\end{itemize}
	\item[2.] Will masks be mandatory to attend lectures in \href{https://www.eecs189.org/about/}{CS189} (as taught by Marvin Zhang, not Jonathan Shewchuk) on March 8th?
	\begin{itemize}
		\item Note that the question is about attending lectures, not discussion sections or office hours.
		\item This question resolves positively if masks are mandatory to attend lectures (whether due to UC Berkeley policy, professor choice, or a vote by the students), unless this \emph{explicitly} contradicts UC Berkeley policy.
	\end{itemize} 
	\item[3.] Will Russia invade Ukraine before End Of Day February 27?
	\begin{itemize}
		\item This question resolves positively if, between February 22nd, 2022 12:00am and February 27th, 2022 11:59pm, representatives of the Government of the Russian Federation announce or acknowledge that Russia has invaded Ukraine, or if any two Permanent Members of the United Nations Security Council announce or acknowledge that the Russian Federation has invaded Ukraine. 

		\item These announcements must be describing events which took place (at least in part) during the same period, from February 22nd, 2022 12:00am to February 27th, 2022 11:59pm. 
		
		\item Areas of Ukraine already occupied (officially or de facto, cf \url{https://en.wikipedia.org/wiki/Temporarily_occupied_territories_of_Ukraine}) by Russia as of January 24th, 2022, will not trigger resolution.
	\end{itemize} 
	\item[4.] How many contestants of Season 2 of \href{https://en.wikipedia.org/wiki/Love_Is_Blind_(TV_series)}{Love Is Blind} will have gotten married on air by the end of S2 E10 on February 25?
\end{enumerate}

For each question, submit a mean and inclusive 80\% confidence interval (or a probability for question 1), as well as an explanation of your reasoning (1-2 paragraphs). \textbf{Please include a copy of your google form responses with your Gradescope submission.} On Gradescope, please also submit the time it took to complete this exercise.

\section*{Longer-term Predictions}

\emph{Expected completion time: 40 minutes}

Register the following predictions. Although they resolve in April, they will also count for the leaderboard. You can also submit them by going to \url{https://forms.gle/hDsAcMvyMP7CJm1C7} and following the form's instructions.


\begin{enumerate}
	\item[1.] Who will Joe Biden replace Eric Lander by as Head of the White House Office of Science and Technology Policy?
	\begin{itemize}
		\item \href{https://www.politico.com/newsletters/morning-tech/2022/02/09/the-race-to-replace-lander-at-ostp-00007114}{This article} names 5 likely nominees: Alondra Nelson, Jane Lubchenco, Jason Matheny, France C\'ordova, Maria Zuber. Please submit a probability for each, and for ``someone else''.
	\end{itemize} 
	\item[2.] Will the California Supreme Court intervene in favour of UC Berkeley in \href{https://www.berkeleyside.org/2022/02/14/uc-berkeley-enrollment-drop-court-of-appeal-ruling}{this case} by March 23?
\end{enumerate}

For each question, submit your probabilities as well as an explanation of your reasoning (1-2 paragraphs). \textbf{Please include a copy of your google form responses with your Gradescope submission} (you don't have to take a screenshot, you can simply copy-paste your reasoning and responses into the file you submit on Gradescope). On Gradescope, please also submit the time it took to complete this exercise.

% \section*{Extra-credit: Predictions}
% \begin{enumerate}
% 	\item[1.] Give an 80$\%$ Confidence Interval for 2/3 of the Upper Bound of the Confidence Intervals provided for this question, as well as an explanation of your reasoning. Once again, submit your prediction and reasoning on \url{TODOurl}, and include a copy of both on your Gradescope submission.
% \end{enumerate}

% \section*{Extra-credit: Question generation}
% For extra credit, you can also generate questions: you will get points if we add your questions to our pool of questions for this year or the following years. Many of our questions so far focused on a few themes like movies, elections or geopolitics, so we're especially interested in questions about other topics! We are interested in 3 kinds of questions:
% \begin{itemize}
% 	\item Calibration exercise questions, such as in the calibration app from \emph{HW1 Deliberate Practice} (1 point).
% 	\item Fermi estimate questions, such as in \emph{HW2 Deliberate Practice: Estimation} (2 points).
% 	\item Forecasting questions like the weekly prediction questions in past homeworks (3 points).
% \end{itemize}


\end{document}
