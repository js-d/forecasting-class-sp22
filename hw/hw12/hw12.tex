\documentclass[11pt]{article}
\usepackage{amsmath,amsfonts,amssymb}
\usepackage{hyperref}
\hypersetup{
	colorlinks=true,
	linkcolor=blue,
	urlcolor=cyan
	}
\usepackage[margin=1in]{geometry}
\usepackage{graphicx}

\setlength{\parindent}{0pc}
\setlength{\parskip}{10pt}

\title{STAT157 HW 12}
\date{April 11, 2022}

\begin{document}

\maketitle

\hfill \textbf{Predictions due Monday, April 18 at 11:59pm}

\hfill \textbf{Gradescope submission due 15 min after}

This is the last homework for the semester. Please use spend the remaining weeks of the semester to work on your final projects.

\section*{Deliberate Practice: Forecasting Pandemics}

\emph{Expected completion time: 90 minutes}

This week's deliberate practice forecasting question comes from our guest lecturer, Juan Cambeiro:
\begin{quote}
	Will Omicron be the most dominant sequenced strain of SARS-CoV-2 on Dec 31, 2022? (\href{https://www.metaculus.com/questions/8880/omicron-dominant-variant-dec-31-2022/}{link} to Metaculus question)
\end{quote}

Generate your own forecast for this question, including a mean and an 80\% confidence interval. Please use at least 4 techniques from previous lectures, from this list:
\begin{itemize}
	\item Fermi estimation
	\item Zeroth and first order forecasting
	\item Base rates and reference classes
	\item Generating ``other" options
	\item Combining forecasts
	\item Approximating with common probability distributions
	\item Prioritizing information and analyzing relative uncertainties
	\item Bayesian aggregation of evidence
\end{itemize}

In your writeup, discuss how you created your forecast. You may write a holistic summary of your thought process, with a small note each time that you used techniques from lecture, or you may write a separate section for each of the techniques you chose, detailing how you incorporated it into your forecast. Your writeup should be at least 300 words. If your write-up would be much longer than 300 words to address every consideration you analyzed, it is okay to summarize some parts instead of providing exhaustive reasoning.

On Gradescope, please also submit the time it took to complete this exercise.

\section*{Predictions}

\emph{Expected completion time: 90 minutes}

Register the following predictions. You can submit them by going to \href{https://docs.google.com/forms/d/e/1FAIpQLSeMnjW4dLgLBFrIVr0n4-KjeakEj00KlN4PCfIRmHAvGj95kQ/viewform?usp=sf_link}{this URL} and following the form's instructions. For these predictions, (and all predictions about the future throughout this class), we encourage you to use external sources -- by googling things, reading news articles, talking to friends who follow politics or music stats, etc.

\begin{enumerate}
	\item Will Omicron be the most dominant sequenced strain of SARS-CoV-2 on Dec 31, 2022? (\href{https://www.metaculus.com/questions/8880/omicron-dominant-variant-dec-31-2022/}{link} to Metaculus question)
	\item When will 100M doses of Moderna's Omicron-specific booster candidate or multi-valent booster candidates be distributed? (\href{https://www.metaculus.com/questions/8767/date-100m-doses-omicron-booster-distributed/}{link} to Metaculus question)
	\item How many confirmed deaths from COVID-19 will be reported globally in 2022? (\href{https://www.metaculus.com/questions/8307/global-confirmed-covid-19-deaths-in-2022/}{link} to Metaculus question)
\end{enumerate}

For each question, submit a mean and inclusive 80\% confidence interval or your probabilities, as well as an explanation of your reasoning (1-2 paragraphs). \textbf{Please include a copy of your google form responses with your Gradescope submission.} On Gradescope, please also submit the time it took to complete this exercise.

\section*{Extra Credit}

You can earn extra credit for doing one or both of the following tasks:
\begin{enumerate}
\item (2 points). Propose a prediction question for a future homework. Your question should have a fully specified resolution criteria, 
      such that it would be clear to the course staff how to resolve it.
\item (2 points). Write down 5 ``calibration-app'' style questions (and answers), for a domain that is not trivia.
\end{enumerate}

For the first part, you will get extra credit if we either use your question on a future homework, 
or like it enough to add it to our question bank for future classes.

For the second part, you will get extra credit if we think they are at least as interesting as the 
questions currently in the calibration app, and well-written enough to be used as a calibration question. 
(We may submit them to the app creator and suggest they include them!)

\end{document}
