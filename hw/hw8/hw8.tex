\documentclass[11pt]{article}
\usepackage{amsmath,amsfonts,amssymb}
\usepackage{hyperref}
\hypersetup{
	colorlinks=true,
	linkcolor=blue,
	urlcolor=cyan
	}
\usepackage[margin=1in]{geometry}
\usepackage{graphicx}

\setlength{\parindent}{0pc}
\setlength{\parskip}{10pt}

\title{STAT157 HW 7}
\date{March 8, 2022}

\begin{document}

\maketitle

\hfill \textbf{Predictions due Monday, March 14 at 11:59pm}

\hfill \textbf{Gradescope submission due 15 min after}


\section*{Deliberate Practice: Forecasting AI}

\emph{Expected completion time: 120 minutes}

In Lectures 14 and 15, we analyzed considerations relevant for an in-depth forecast about the future of AI. In this deliberate practice, you will do your own forecast by synthesizing the information presented in lecture, and optionally doing a bit of your own research. If you include in your writeup a data source that the staff isn't currently aware of, and that updates the staff's own forecasts, you can get up to 3 points of extra credit.


The forecast addressed in class was:
\begin{quote}
	In what year will AI systems be able to perform essentially all tasks that humans can do?
\end{quote}
Generate your own forecast for this question, including a mean and an 80\% confidence interval. Please use at least 4 techniques from previous lectures, from this list:
\begin{itemize}
	\item Fermi estimation
	\item Zeroth and first order forecasting
	\item Base rates and reference classes
	\item Generating ``other" options
	\item Combining forecasts
	\item Approximating with common probability distributions
	\item Prioritizing information and analyzing relative uncertainties
	\item Bayesian aggregation of evidence
\end{itemize}

In your writeup, discuss how you created your forecast. You may write a holistic summary of your thought process, with a small note each time that you used techniques from lecture, or you may write a separate section for each of the techniques you chose, detailing how you incorporated it into your forecast. Your writeup should be at least 500 words.




On Gradescope, please also submit the time it took to complete this exercise.


\section*{Predictions}

\emph{Expected completion time: 120 minutes}

Register the following predictions. You can submit them by going to \href{https://docs.google.com/forms/d/e/1FAIpQLSceQC49WmqDIG0Y9E-O7aTNcfS-hGP_HMg5Qem4W1F5-pt8QQ/viewform?usp=sf_link}{this URL} and following the form's instructions. For these predictions, (and all predictions about the future throughout this class), we encourage you to use external sources -- by googling things, reading news articles, talking to friends who follow politics or music stats, etc.

\begin{enumerate}
	\item Will Congress pass a bill banning oil imports from Russia by 11:59pm on April 15?
	\item Will the Lapsus$\$$ hacking group release NVIDIA Verilog files on their Telegram channel between March 14, 11:59pm and March 28, 11:59pm?
	\item What will be the total number of acres in California burned by wildfires in 2022 by April 15?
	\begin{itemize}
		\item This question will resolve based on the data in \href{https://www.fire.ca.gov/incidents/2022/}{this database} of incidents.
	\end{itemize}
	\item Will there still be construction for the \href{https://www.berkeleyside.org/2021/07/07/milvia-street-berkeley-construction-bike-lanes}{Milvia bikeway} between Hearst and University on April 15?
	\begin{itemize}
		\item We will say there is no more construction if (1) there are at most 5 of \href{https://drive.google.com/file/d/1lnhlF9xAOJ5u1CbSyMmXPxyTUJtdXxue/view?usp=sharing}{these orange bollards} left between Hearst and University, and (2) there is no longer \href{https://drive.google.com/file/d/12rqYnh4DPwEApKft4acDz9lzGxiLPkzr/view?usp=sharing}{this ``Road Works Ahead'' sign} on the angle of Berkeley Way and Milvia Street.
	\end{itemize}
	\item (Extra-credit, 2 points) Will Kyiv to fall to Russian forces by April 2022?
	\begin{itemize}
		\item The question will resolve like \href{https://www.metaculus.com/questions/9939/kyiv-to-fall-to-russian-forces-by-april-2022/}{this Metaculus question}.
	\end{itemize}
\end{enumerate}

For each question, submit a mean and inclusive 80\% confidence interval or your probabilities, as well as an explanation of your reasoning (1-2 paragraphs). \textbf{Please include a copy of your google form responses with your Gradescope submission.} On Gradescope, please also submit the time it took to complete this exercise.

\section*{Extra Credit}

You can earn extra credit for doing one or both of the following tasks:
\begin{enumerate}
\item (2 points). Propose a prediction question for a future homework. Your question should have a fully specified resolution criteria, 
      such that it would be clear to the course staff how to resolve it.
\item (2 points). Write down 5 ``calibration-app'' style questions (and answers), for a domain that is not trivia.
\end{enumerate}

For the first part, you will get extra credit if we either use your question on a future homework, 
or like it enough to add it to our question bank for future classes.

For the second part, you will get extra credit if we think they are at least as interesting as the 
questions currently in the calibration app, and well-written enough to be used as a calibration question. 
(We may submit them to the app creator and suggest they include them!)

\end{document}
