\documentclass[11pt]{article}
\usepackage{amsmath,amsfonts,amssymb}
\usepackage{hyperref}
\hypersetup{
	colorlinks=true,
	linkcolor=blue,
	urlcolor=cyan
	}
\usepackage[margin=1in]{geometry}
\usepackage{graphicx}

\setlength{\parindent}{0pc}
\setlength{\parskip}{10pt}

\title{STAT157 HW 11}
\date{April 4, 2022}

\begin{document}

\maketitle

\hfill \textbf{Predictions due Monday, April 11 at 11:59pm}

\hfill \textbf{Gradescope submission due 15 min after}

This week's homework assignment is again shorter than usual, with only the Predictions section and optional extra credit. Please use the time that you would have spent on a normal length homework to work on your final projects.


\section*{Predictions}

\emph{Expected completion time: 90 minutes}

Register the following predictions. You can submit them by going to \href{TODO}{this URL} and following the form's instructions. For these predictions, (and all predictions about the future throughout this class), we encourage you to use external sources -- by googling things, reading news articles, talking to friends who follow politics or music stats, etc.

\begin{enumerate}
	\item 
\end{enumerate}

For each question, submit a mean and inclusive 80\% confidence interval or your probabilities, as well as an explanation of your reasoning (1-2 paragraphs). \textbf{Please include a copy of your google form responses with your Gradescope submission.} On Gradescope, please also submit the time it took to complete this exercise.

\section*{Extra Credit}

You can earn extra credit for doing one or both of the following tasks:
\begin{enumerate}
\item (2 points). Propose a prediction question for a future homework. Your question should have a fully specified resolution criteria, 
      such that it would be clear to the course staff how to resolve it.
\item (2 points). Write down 5 ``calibration-app'' style questions (and answers), for a domain that is not trivia.
\end{enumerate}

For the first part, you will get extra credit if we either use your question on a future homework, 
or like it enough to add it to our question bank for future classes.

For the second part, you will get extra credit if we think they are at least as interesting as the 
questions currently in the calibration app, and well-written enough to be used as a calibration question. 
(We may submit them to the app creator and suggest they include them!)

\end{document}
