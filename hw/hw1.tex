\documentclass[11pt]{article}
\usepackage{amsmath,amsfonts,amssymb}
\usepackage{hyperref}
\usepackage[margin=1in]{geometry}
\setlength{\parindent}{0pc}
\setlength{\parskip}{10pt}

\title{STAT157 HW 1}
\date{Jan 19, 2022}

\begin{document}

\maketitle

\hfill \textbf{Due Monday, January 24 at 7pm}

\section*{Logistics Setup}

\emph{Expected completion time: 10 minutes}

Install a time-tracking app as described in the ``Predictions'' section, and submit a screenshot or photo showing it has been 
installed.

\section*{Deliberate Practice}

\emph{Expected completion time: 80 minutes}

Go to \url{https://www.openphilanthropy.org/calibration} and sign in (either create an account or use 
Gmail/Facebook). Click ``Home'' and complete the following exercises in the app:

\begin{itemize}

\item Confidence intervals (level 80\%) -- at least 30 questions
\item 30 questions each from two of the other categories (i.e., two out of PolitiFact, Correlations, City Populations, Math, and Trivia).

\end{itemize}

(Of course, feel free to answer more questions if you would like!)

At the end, go to the ``Results'' tab and take screenshots showing that 
you completed at least 30 questions in each category, as well as your calibration 
performance (the chart that appears below the results).

\section*{Predictions}

\emph{Expected completion time: 35 minutes}

Register the following predictions. You can submit them by going to 
\url{https://forms.gle/Z9S48GfsTgjA2bHW9} and following the form's instructions.

\begin{enumerate}
\item[0.] Pick a website, application, or software of your choice. Predict how much time you will spend on it between 12:00am Tuesday January 25th and 11:59pm Sunday January 30th. You should track this {\bf using a time-tracking app} such as \href{https://support.apple.com/en-us/HT210387}{Screen Time} (if you have macOS Catalina or later), \href{https://www.rescuetime.com/}{Rescue Time}, Digital Wellbeing (Android), or any another tracker you prefer.\footnote{Note that some desktop apps record screen time even when an app is open in the background; you might want to be careful about this when making predictions or choosing the application.}

\item[1.] How many people will be in the Zoom lecture at 11:15am on Wednesday, January 26th?

      This will resolve based on a count conducted by the course staff.

\item[2.] How many dollars will the movie Scream gross in the US on the weekend of Jan 28-30? 

      This will resolve according to \href{https://www.boxofficemojo.com/release/rl307200769/weekend/?ref_=bo_rl_tab#tabs}{Box Office Mojo}.

\item[3.] Will Scream be the highest-grossing movie in the US on the weekend of Jan 28-30?

      This will also resolve according to \href{https://www.boxofficemojo.com/weekend/?ref_=bo_nb_wey_secondarytab}{Box Office Mojo}.

\end{enumerate}
 
For each question, submit a mean and inclusive 80\% confidence interval (or a probability for question 3), 
as well as an explanation of your reasoning (~1-2 paragraphs).

For questions 1-3, your prediction (but not the explanation) will appear on the public leaderboard. 
Question 0 will remain private and not count towards the leaderboard.

\end{document}
