\documentclass[11pt]{article}
\usepackage{amsmath,amsfonts,amssymb}
\usepackage{hyperref}
\hypersetup{
	colorlinks=true,
	linkcolor=blue,
	urlcolor=cyan
	}
\usepackage[margin=1in]{geometry}
\setlength{\parindent}{0pc}
\setlength{\parskip}{10pt}

\title{STAT157 HW 2}
\date{Jan 24, 2022}

\begin{document}

\maketitle

\hfill \textbf{Due Monday, January 31 at 11:59pm}

\section*{Deliberate Practice: Calibration}

\emph{Expected completion time: 30 minutes}

Go to \url{https://www.openphilanthropy.org/calibration} and complete 30 questions under the Confidence Intervals (level 60\%) category.

At the end, go to the ``Results'' tab and take screenshots showing that 
you completed at least 30 questions and your calibration 
performance (the chart that appears below the results).

\section*{Deliberate Practice: Estimation}

\emph{Expected completion time: 60 minutes}


For each of the following questions, estimate the answer without looking things up, then look up the answer and calculate the relative error of your estimate\footnote{Note that some of these questions still leave some ambiguity depending on specific definitions, and different Google search results might give slightly different answers!}. For each quantity, we provided one link with a reasonable-seeming answer, which you should use as the ``official'' answer when you calculate the error. We recommend spending around 5 minutes on each estimation question. You will be graded on your thinking process, not your accuracy.

In addition to the questions below, devise two estimation questions of your own, for quantities you are interested in. Again, estimate the answer without looking things up, then look up a reasonable answer and compute your relative error.

\begin{enumerate}
	\item How many cattle are in the world in 2021? 
	\item What's the length of the SF Bay Bridge, in meters? 
	\item How heavy, in tons, is the Titanic? 
	\item How many paid subscribers of Netflix are there as of Q3 of 2021?
	\item How many statistics and biostatistics degrees (Bachelors plus Masters plus PhD) were given in the U.S. in 2020? 
	\item What was the GDP of Mexico in 2020? 
	\item How many musicians are employed in the U.S.?
	\item What's the depth of the deepest hole, in meters, that humans have dug? 
\end{enumerate}


Here are our links that contain answers:
\begin{enumerate}
\item \url{https://www.statista.com/statistics/263979/global-cattle-population-since-1990/}
\item \url{https://en.wikipedia.org/wiki/San_Francisco%E2%80%93Oakland_Bay_Bridge}
\item \url{https://en.wikipedia.org/wiki/Titanic}
\item \url{https://www.insiderintelligence.com/insights/netflix-subscribers/}
\item \url{https://magazine.amstat.org/blog/2021/10/01/undergrad-stats-degrees-up/}
\item \url{https://countryeconomy.com/gdp/mexico}
\item \url{https://www.zippia.com/musician-jobs/demographics/}
\item \url{https://en.wikipedia.org/wiki/Kola_Superdeep_Borehole}
\end{enumerate}

On Gradescope, for each of the \textbf{10 questions (8 provided plus 2 created by you)}, submit your estimate, an explanation of your estimate, and your relative error.


\section*{Lab}

\emph{Expected completion time: 90 minutes}

\href{https://datahub.berkeley.edu/hub/user-redirect/git-pull?repo=https%3A%2F%2Fgithub.com%2Fjs-d%2Fstat-157-260-website&urlpath=tree%2Fstat-157-260-website%2Fhw%2Fhw2%2Fhw2_lab.ipynb&branch=main}{Link to Jupyter notebook.} 

Please follow the instructions in the notebook to print out your code and answers and submit to Gradescope. You may use languages other than Python, although we will generally be providing starter code in Python.



\section*{Predictions}

\emph{Expected completion time: 60 minutes}

Register the following predictions. You can submit them by going to \url{https://forms.gle/o89xw3s2iEr252KK6} and following the form's instructions. For these predictions, (and all predictions about the future throughout this class), we encourage you to use external sources -- by googling things, reading news articles, talking to friends who follow politics or music stats, etc.

\begin{enumerate}
	\item[0.] Pick the same website, application, or software you chose last week, and predict how much time you will spend on it between 12:00am Tuesday February 1st and 11:59pm Sunday February 6th., as measured by your time-tracking app.
	
	\item[1.] How many percents of the vote will Figueres get in the first round of the Costa Rican presidential \href{https://en.wikipedia.org/wiki/2022_Costa_Rican_general_election}{election} on Feb 6?
	
	\item[2.] Will Adele's \textit{Easy on me} top the \href{https://www.billboard.com/charts/hot-100/}{Billboard Hot 100} on the week of Feb 5?
 
	\item[3.] Will Russia invade Ukraine between Feb 1 and Feb 7?
	
	\item[4.] How many new cases of Covid-19 will be reported on the \href{https://sf.gov/data/covid-19-cases-and-deaths}{San Francisco Covid dashboard} for Tuesday Feb 1? (The daily case count has a lag of about a week before the numbers are reliable, so this forecast will resolve on Feb 8, when we will look at the case count listed for Feb 1.) 
	
	
\end{enumerate}

For each question, submit a mean and inclusive 80\% confidence interval (or a probability for questions 2 and 3), as well as an explanation of your reasoning (~1-2 paragraphs). \textbf{Please include a copy of your google form responses with your Gradescope submission.}


\end{document}